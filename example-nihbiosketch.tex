%!TEX TS-program = xelatex
\documentclass{nihbiosketch}

%------------------------------------------------------------------------------
\name{Brozdowski, Chris}
\eracommons{CBrozdowski}
\position{Neuroscience Data Engineer}

\begin{document}
%------------------------------------------------------------------------------

\begin{education}
University of Connecticut, Storrs       & B.S         & 05/2013 & Cognitive Science \\
University of California, San Diego \& San Diego State University
                                        & Ph.D.        & 08/2018 & Language and Communicative Disorders \\
RWTH Aachen University, Aachen, Germany & Postdoctoral & 10/2019 & Mechanolinguistics of Sign \\
Vanderbilt University, Nashville        & Postdoctoral & 10/2021 & Neuroscience of Literacy \\
\end{education}


\section{Personal Statement}

\begin{statement}
In my career, I've cultivated two domains of expertise. First, my projects across pre-
and postdoctoral work examined the cognitive science topics in Deaf and Hard of
Hearing populations through behavioral and neuroimaging studies. My
predoctoral work investigated the cognitive impacts of native American Sign Language
exposure for Deaf adults, including spatial cognition and predictive processing. In my
postdoctoral work, these questions extended to gesture production and the development of
literacy among Hard of Hearing children.
My various positions have fostered a second expertise in the research
infrastructure used to standardize processes, expedite development, and promote
collaborative uses of data. I developed tools to automate gesture transcription, stimulus
presentation, testing score retrieval, etc. In each case, I focused on 
usability from the perspective of a new lab member. At DataJoint, I've
continued this emphasis on research efficiency through approachable tools via 
DataJoint Elements, a collection of open-source packages to help scientists build
transparent and reproducible computational workflows. With Element DeepLabCut, I
built a pipeline for managing pose estimation models and results, while
maintaining an open and productive dialog with our userbase. My research background
gives me a strong understanding of how scientists with little technical training can
benefit from automation. The technical, communication, and mentoring skills I've required
over the course of my career allow me to generate user-friendly tools, successfully
contributing to the ongoing research project.

\begin{enumerate}

\item Emmorey, K., \textbf{Brozdowski, C.}, \& McCullough, S. (2021). The neural
        correlates for spatial language: Perspective-dependent and -independent
        relationships in American Sign Language and spoken English,
        \textit{Brain and Language}, 223, 105044.

\item \textbf{Brozdowski, C.}, \& Booth, J. R. (2021). Reading skill correlates in
        frontal cortex during semantic and phonological processing.

\item \textbf{Brozdowski, C.} \& Emmorey, K. (2020) Shadowing in the manual modality.
        \textit{Acta Psychologica}, 108, 103092.

\item \textbf{Brozdowski, C.}, Secora, K., \& Emmorey, K. (2019). Assessing the
        Comprehension of Spatial Perspectives in ASL Classifier Constructions.
        \textit{The Journal of Deaf Studies and Deaf Education} 24(3), 214-222.

\end{enumerate}

\end{statement}

%------------------------------------------------------------------------------
\section{Positions, Scientific Appointments, and Honors}

\subsection*{Positions and Employment}
\begin{datetbl}
2010--2013 & Research Assistant, Various Labs, University of Connecticut, Storrs, CT\\
2013--2018 & Graduate Researcher, SLHS, San Diego State University, San Diego, CA \\
2018--2019 & Researcher and Lecturer, RWTH Aachen University, Aachen, Germany  \\
2019--2021 & Researcher, Dept of Psych. \& Human Dev't, Vanderbilt University, Nashville, TN \\
2021--     & Neuroscience Data Engineer, DataJoint, Houston, TX \\
\end{datetbl}

\subsection*{Honors and Awards}
\begin{datetbl}
2012       & University of Connecticut Summer Undergraduate Research Fund Recipient, Storrs, CT \\
2013       & Honorable Mention, National Science Foundation Graduate Research Fellowship Program \\
2016       & Travel Award, Theoretical Issues in Sign Language Research, Melbourne, ASTL. \\
2018       & Travel Award, University of Michigan NIH Training Course for fMRI, Ann Arbor, MI \\
\end{datetbl}

%------------------------------------------------------------------------------


\section{Contribution to Science}

\begin{enumerate}
\item Scientists often find idiosyncratic solutions to meet the needs of unique,
innovative experiments. Bespoke time-consuming solutions sometimes reinvent
well-validated industry standards, impeding reproducibility, collaboration, and open
science. To address these issues, the DataJoint team is developing a browser-based
cloud-computation platform (i.e., Works) that would allow researchers to upload data
into well-validated table architectures and automate common preprocessing steps. Works
would cut down on individual development and processing time, as well as open projects
to collaborators for easy replication and reanalysis. This centralized vision for data
managment and workflow automation relies on working with (a) experts in the respective
modalities to converge on useful pipelines that can accomodate variability across
experiments, (b) other tool developers to ensure interoperability and (c) end-users to
ensure that the finished product is both managable and transparent for researchers with
little computer science training.

\item Deafness and sign language exposure, both independently and jointly, reshape how
 individuals engage with the world. Spatial cognition serves as a clear example. To
 describe spatial relationships, American Sign Lanugage users use the space in front of
 themselves in an analog mapping to real world referents. In these Classifier
 Constructions, it's common to describe the world from one's own perspective. For the
 interlocutor, accomodating the signer's perspective is slower (Brozdowski, Secora, \&
 Emmorey, 2019) and entails increased neural activation in the superior pareital
 lobule (Emmorey, Brozdowski, \& McCullough, 2021).

\begin{enumerate}

\item Emmorey, K., \textbf{Brozdowski, C.}, \& McCullough, S. (2021). The neural
        correlates for spatial language: Perspective-dependent and -independent
        relationships in American Sign Language and spoken English,
        \textit{Brain and Language}, 223, 105044.

\item \textbf{Brozdowski, C.}, Secora, K., \& Emmorey, K. (2019). Assessing the
        Comprehension of Spatial Perspectives in ASL Classifier Constructions.
        \textit{The Journal of Deaf Studies and Deaf Education} 24(3), 214-222.

\end{enumerate}
\item Prevailing theories of language comprehension and literacy development are often
 uniqely supported with evidence from hearing spoken language users. By extending
 research to include signers and deaf individuals, researchers can test the boundaries
 of these theories. For example, evidence suggests that spoken language users rely on
 motor simulation (i.e., subvocal immitation) for immediate comprehension. Under
 similar circumstances in the manual modality, nonsigners, but not signers,
 demonstrated key markers of motor simulation (Brozdowski \& Emmorey, 2020). In the
 domain of literacy, many theories ephasize the interaction between orthography,
 phonology, and semantics. This body of work describes the reading acqusition process
 as a shift from reliance on ortho-phonological to ortho-semantic pathways
 (Brozdowski \& Booth, 2021). Ongoing data collection will
 examine the impacts of varying levels of deafness and sign language exposure in a
 logitudinal neuroimaging study.

\begin{enumerate}   

\item \textbf{Brozdowski, C.}, \& Booth, J. R. (2021). Reading skill correlates in
        frontal cortex during semantic and phonological processing.

\item \textbf{Brozdowski, C.} \& Emmorey, K. (2020) Shadowing in the manual modality.
        \textit{Acta Psychologica}, 108, 103092.

\end{enumerate}

\end{enumerate}

%------------------------------------------------------------------------------

\section{Additional Information: Research Support and/or Scholastic Performance}

\end{document}
