%!TEX TS-program = xelatex
\documentclass{nihbiosketch}

% \usepackage{draftwatermark}  % delete this in your document!
% \SetWatermarkText{Sample}    % delete this in your document!
% \SetWatermarkLightness{0.9}  % delete this in your document!

%------------------------------------------------------------------------------

\name{Brozdowski, Chris}
\eracommons{CBrozdowski}
\position{Neuroscience Data Engineer}

\begin{document}
%------------------------------------------------------------------------------

\begin{education}
University of Connecticut, Storrs       & B.S         & 05/2013 & Cognitive Science \\
University of California, San Diego \& San Diego State University
                                        & Ph.D.        & 08/2018 & Language and Communicative Disorders \\
RWTH Aachen University, Aachen, Germany & Postdoctoral & 10/2019 & Mechanolinguistics of Sign \\
Vanderbilt University, Nashville        & Postdoctoral & 10/2021  & Neuroscience of Literacy \\
\end{education}


\section{Personal Statement}

\begin{statement}
In my career, I've cultivated two domains of expertise. First, my projects across pre-
and postdoctoral work have examined the cognitive science topics in Deaf and Hard of
Hearing populations through behavioral, motion capture, and neuroimaging studies. My
predoctoral work investigated the cognitive impacts of native American Sign Language
exposure for Deaf adults, including spatial cognition and predictive processing. In my
postdoctoral work, these questions extended to gesture production among German signers
and the development of literacy among Deaf and Hard of Hearing children, while also
mentoring research assistants. My various positions have also fostered expertise in the
research infrastructure used to standardize processes, expedite development, and promote
collaborate uses of data. I developed tools to automate gesture transcription, stimulus
presentation, standardized testing score retrieval, etc. In each case, I focused on the
usability of the tool from the perspective of a new lab member. At DataJoint, I've
continued this emphasis on research efficiency through approachable tools via the
DataJoint Elements, a collection of open source Python packages to help scientists build
transparent and reproducible computational workflows. My research background gives me a
strong understanding of how scientists with little technical training can benefit from
automation. The technical, communication and mentoring skills I've required over the
course of my career allow me to generate such user-friendly tools, successfully
contributing to the ongoing research project.

\begin{enumerate}

\item Emmorey, K., \textbf{Brozdowski, C.}, \& McCullough, S. (2021). The neural
        correlates for spatial language: Perspective-dependent and -independent
        relationships in American Sign Language and spoken English,
        \textit{Brain and Language}, 223, 105044.

\item \textbf{Brozdowski, C.}, \& Booth, J. R. (2021). Reading skill correlates in
        frontal cortex during semantic and phonological processing.

\item \textbf{Brozdowski, C.} \& Emmorey, K. (2020) Shadowing in the manual modality.
        \textit{Acta Psychologica}, 108, 103092.

\item \textbf{Brozdowski, C.}, Secora, K., \& Emmorey, K. (2019). Assessing the
        Comprehension of Spatial Perspectives in ASL Classifier Constructions.
        \textit{The Journal of Deaf Studies and Deaf Education} 24(3), 214-222.

\end{enumerate}

\end{statement}

%------------------------------------------------------------------------------
\section{Positions, Scientific Appointments, and Honors}

\subsection*{Positions and Employment}
\begin{datetbl}
2010--2013 & Research Assistant, Various Labs, University of Connecticut, Storrs, CT\\
2013--2018 & Graduate Researcher, SLHS, San Diego State University, San Diego, CA \\
2018--2019 & Researcher and Lecturer, RWTH Aachen University, Aachen, Germany  \\
2019--2021 & Researcher, Dept of Psych. \& Human Dev't, Vanderbilt University, Nashville, TN \\
2021--     & Neuroscience Data Engineer, DataJoint, Houston, TX \\
\end{datetbl}

\subsection*{Honors and Awards}
\begin{datetbl}
2012       & University of Connecticut Summer Undergraduate Research Fund Recipient, Storrs, CT \\
2013       & Honorable Mention, National Science Foundation Graduate Research Fellowship Program \\
2016       & Travel Award, Theoretical Issues in Sign Language Research, Melbourne, ASTL. \\
2018       & Travel Award, University of Michigan NIH Training Course for fMRI, Ann Arbor, MI \\
\end{datetbl}

%------------------------------------------------------------------------------


\section{Contribution to Science}

\begin{enumerate}
\item Scientists are accustomed to finding idiosyncratic solutions to meet the needs of
 unique and innovative experiments. Unfortunately, this individuality can sometimes
 hinder interoperability across labs, impeding collaboration and open science. Bespoke
 time-consuming solutions sometimes reinvent well-validated industry standards,
 impeding reproducibility. Open and reproducible science requires not only sharing the
 data, but transparent practices for rerunning analyses. To address these issues,
 DataJoint Python and MATLAB APIs make the industry-standard of relational databases
 accessible to scientists with little training in data engineering (Yatsenko et al.,
 2015), and(b) DataJoint Elements provide validated and transparent data pipelines for
 common modalities (Yatsenko et al., 2021). DataJoint Works allows research to upload
 data to a cloud-based service that automatically populates the Elements for either
 array electrophysiology or calcium imaging, and sends notifications when the data are
 ready for analyses specific to the experiment.

\begin{enumerate}

\item Yatsenko D., Reimer J., Ecker A.S., Walker E.Y., Sinz F., Berens P., et al.
        DataJoint: managing big scientific data using MATLAB or Python.
        \textit{bioRxiv}. 2015 Jan 1:031658. doi: https://doi.org/10.1101/031658

\item Yatsenko D., Nguyen T., Shen S., Gunalan K., Turner C.A., Guzman R., et al.
        DataJoint Elements: Data Workflows for Neurophysiology. \textit{bioRxiv}.
        2021 Jan 1. doi: https://doi.org/10.1101/2021.03.30.437358

\end{enumerate}
\item Deafness and sign language exposure, both independently and jointly, reshape how
 individuals engage with the world. Spatial cognition serves as a clear example. To
 describe spatial relationships, American Sign Lanugage users use the space in front of
 themselves in an analog mapping to real world referents. In these Classifier
 Constructions, it's common to describe the world from one's own perspective. For the
 interlocutor, accomodating the signer's perspective is slower(Brozdowski, Secora, \&
 Emmorey, 2019) and entails increased neural activation in the superior pareital
 lobule (Emmorey, Brozdowski, \& McCullough, 2021).

\begin{enumerate}

\item Emmorey, K., \textbf{Brozdowski, C.}, \& McCullough, S. (2021). The neural
        correlates for spatial language: Perspective-dependent and -independent
        relationships in American Sign Language and spoken English,
        \textit{Brain and Language}, 223, 105044.

\item \textbf{Brozdowski, C.}, Secora, K., \& Emmorey, K. (2019). Assessing the
        Comprehension of Spatial Perspectives in ASL Classifier Constructions.
        \textit{The Journal of Deaf Studies and Deaf Education} 24(3), 214-222.

\end{enumerate}
% THIS IS KABI's
\item Prevailing theories of language comprehension and literacy development are often
 uniqely supported with evidence from hearing spoken language users. By extending
 research to include signers and deaf individuals, researchers can test the boundaries
 of these theories. For example, evidence suggests that spoken language users rely on
 motor simulation (i.e., subvocal immitation) for immediate comprehension. Under
 similar circumstances in the manual modality, nonsigners, but not signers,
 demonstrated key markers of motor simulation (Brozdowski \& Emmorey, 2020). In the
 domain of literacy, many theories ephasize the interaction between orthography,
 phonology, and semantics. This body of work describes the reading acqusition process
 as a shift from reliance on ortho-phonological to ortho-semantic pathways
 (Brozdowski \& Booth, 2021). Ongoing data collection at my previous position will
 examine the impacts of varying levels of deafness and sign language exposure in a
 logitudinal neuroimaging study.

\begin{enumerate}   

\item \textbf{Brozdowski, C.}, \& Booth, J. R. (2021). Reading skill correlates in
        frontal cortex during semantic and phonological processing.

\item \textbf{Brozdowski, C.} \& Emmorey, K. (2020) Shadowing in the manual modality.
        \textit{Acta Psychologica}, 108, 103092.

\end{enumerate}

\end{enumerate}

%------------------------------------------------------------------------------

\section{Additional Information: Research Support and/or Scholastic Performance}

\end{document}
