%!TEX TS-program = xelatex
\documentclass{nihbiosketch}

\usepackage{draftwatermark}  % delete this in your document!
\SetWatermarkText{Sample}    % delete this in your document!
\SetWatermarkLightness{0.9}  % delete this in your document!

%------------------------------------------------------------------------------

\name{Robertson-Chang, Leilani}
\eracommons{RobertsonL}
\position{Postdoctoral Researcher}

\begin{document}

%------------------------------------------------------------------------------

\begin{education}
Swarthmore College & BS & 08/1995 & 05/1999 & Engineering \\
UC San Diego & PHD & 08/2001 & 09/2007 & Molecular Biology \\
Michigan State University & NIH training grant & 09/2007 & present & Bioinformatics/Immunology \\
\end{education}

%------------------------------------------------------------------------------
\section{Personal Statement}

\begin{statement}
My long term research interests involve the development of a comprehensive understanding of key developmental pathways and how alterations in gene expression contribute to human disease. My academic training and research experience have provided me with an excellent background in multiple biological disciplines including molecular biology, microbiology, biochemistry, and genetics. As an undergraduate, I was able to conduct research with Dr. Xavier Factor on the mechanisms of action of a new class of antibiotics. As a predoctoral student with Dr. Tanti Auguri, my research focused on the regulation of transcription in yeast, and I gained expertise in the isolation and biochemical characterization of transcription complexes. I developed a novel protocol for the purification for components of large transcription complexes. I was first author of the initial description of the Most Novel Complex. A subsequent first author publication challenged a key paradigm of transcription elongation and was a featured article in a major journal. During my undergraduate and graduate careers, I received several academic and teaching awards. For my postdoctoral training, I will continue to build on my previous training in transcriptional controls by moving into a mammalian system that will allow me to address additional questions regarding the regulation of differentiation and development. My sponsor Dr. I.M. Creative is an internationally recognized leader in the transcription/chromatin field and has an extensive record for training postdoctoral fellows. The proposed research will provide me with new conceptual and technical training in developmental biology and whole genome analysis. In addition, the proposed training plan outlines a set of career development activities and workshops -- e.g. grant writing, public speaking, lab management, and mentoring students -- designed to enhance my ability to be an independent investigator. My choice of sponsor, research project, and training will give me a solid foundation to reach my goal of studying developmental diseases in man. During my second postdoctoral year in Dr. Creative's lab my father had a severe stroke that eventually ended his life. I was out of the lab for six months dealing with my father's incapacitating illness and end-of-life issues. This hiatus in training reduced my scientific productivity.

\begin{enumerate}
\item
Robertson-Chang L, Schneider K, Chen M, Auguri T. Rapid isolation and characterization of the most novel transcription complex in Saccharomyces cerevisiae and its role in transcription elongation. CSHL Meeting on Mechanisms of Eukaryotic Transcription; 2009 August; Cold Spring Harbor, NY.
\item
Robertson-Chang L, Schneider K, Chen M, Auguri T. Rapid isolation and characterization of the most novel transcription complex in Saccharomyces cerevisiae and its role in transcription elongation. Cell. 2006; 128:770.
\item
Robertson-Chang L, Auguri T. A tandem affinity purification tag approach allows for isolation of interacting proteins in Saccharomyces cerevisiae. Proceedings of the National Academy of Sciences of the United States of America. 2004; 98:151.
\end{enumerate}

\end{statement}

%------------------------------------------------------------------------------
\section{Positions, Scientific Appointments, and Honors}

\subsection*{Positions and Employment}
\begin{datetbl}
2008 --      & Postdoctoral Researcher, Michigan State University \\
2007 -- 2007 & Postdoctoral Researcher, UC San Diego \\
1999 -- 2001 & Engineer, The IBeam Group \\
\end{datetbl}

\subsection*{Other Experience and Professional Memberships}
\begin{datetbl}
2002 --      & Member, National Society for Bioinformatics and Biotechnology \\
2000 --      & Member, Association for Women in Science \\
1997 --      & Member, Sigma Xi \\
\end{datetbl}

\subsection*{Honors}
\begin{datetbl}
2002 -- 2005 & Predoctoral Fellowship for Minorities, Ford Foundation \\
2001         & STAR award for public service in engineering, The IBeam Group \\
1999         & B.S. awarded with high honors, Swarthmore College \\
1999         & Paula F. Laufenberg award for best senior project in the Department of Engineering, Swarthmore College \\
1995 -- 1999 & Scholarship, National Merit Scholarship Program \\
1995 -- 1997 & Scholarship, Daughters of Hawaii Society \\
\end{datetbl}


%------------------------------------------------------------------------------
\section{Contribution to Science}

\begin{enumerate}

\item
{\bfseries Early Career:}
My early career contributions were focused on applying my knowledge of structural engineering to improving the design and integrity of tensile structures. More specifically, I worked with a team of engineers at the IBeam Group to develop concrete with a higher tensile strength that could be utilized in large structures such as suspension bridges. My particular role in the project was to identify candidate polymers, determine the ultimate tensile strength of these polymers, and make recommendations as to which polymer would afford concrete the most structural integrity under various stresses.

\begin{enumerate}
\item
Lorentson C, Robertson-Chang L, Sauer N, Mehta S. Use of high-tensile concrete in cantilevered structures. J Applied Engineering. 2000; 63:413.
\item
Robertson-Chang L, Janessa AJ. Redesigning the Golden Gate bridge. National Undergraduate Symposium on Science and Engineering; 1998; Baltimore, MD. c1998.
\end{enumerate}

\item
{\bfseries Graduate Career:}
My graduate research contributions focused on transcriptional gene regulation in Saccharomyces cerevisiae. Results from my research were highly relevant as they provided new details into the workings of complex biological systems, and allowed for further extrapolations into the development of certain diseases and their progression. I originally developed a novel protocol for the purification for components of large protein complexes. A subsequent publication, in which I isolated and characterized a long sought after transcription complex, challenged a key paradigm of transcription elongation and was a featured article in a major journal.

\begin{enumerate}
\item
Robertson-Chang L, Schneider K, Chen M, Auguri T. Rapid isolation and characterization of the most novel transcription complex in Saccharomyces cerevisiae and its role in transcription elongation. CSHL Meeting on Mechanisms of Eukaryotic Transcription; 2009 August; Cold Spring Harbor, NY.
\item
Robertson-Chang L, Schneider K, Chen M, Auguri T. Rapid isolation and characterization of the most novel transcription complex in Saccharomyces cerevisiae and its role in transcription elongation. Cell. 2006; 128:770. 
\item
Robertson-Chang L, Auguri T. A tandem affinity purification tag approach allows for isolation of interacting proteins in Saccharomyces cerevisiae. Yeast Genetics and Molecular Biology Meeting; 2004 September; Seattle, WA. 
\item
Robertson-Chang L, Auguri T. A tandem affinity purification tag approach allows for isolation of interacting proteins in Saccharomyces cerevisiae. Proceedings of the National Academy of Sciences of the United States of America. 2004; 98:151.
\end{enumerate}

\item
{\bfseries Postdoctoral Career:}
As a postdoctoral fellow, my research has provided a compelling link between mutations arising in stress response proteins and the development of various autoimmune diseases in humans. Previous studies have shown dysregulation in the innate immune response lead to autoimmune diseases in humans. A few Rtc homologues have now been identified in humans and appear to play a role in the regulation of genes in the innate immune response. My research is focused on the transcriptional regulator Rtc from Drosophila melanogastor.

\begin{enumerate}
\item
Robertson-Chang L, Cescaloo Q, Murray GC. Structural analysis of Drosophila Rtc. Nature. Forthcoming; 
\item
Robertson-Chang L, Yager LN, Murray GC. Rtc is an essential component of the Drosophila innate immune response. Genetics. 2007; 145:884.
\item
Yao M, Dionne CF, Robertson-Chang L, Murray GC. Up-regulation of Drosophila innate immunity genes in response to stress. Science (New York, N.Y.). 2007; 304:1754.
\item
Robertson-Chang L, Murray GC. Stress, flies, and videotape: the Drosophila stress response. Annual review of physiology. 2006; 346:223.
\end{enumerate}

\end{enumerate}

\subsubsection*{Complete List of Published Work in MyBibliography:} 
\url{http://www.ncbi.nlm.nih.gov/sites/myncbi/collections/public/1tay8xsxteXIw5R2StTcjhq5X}


%------------------------------------------------------------------------------
\section{Additional Information: Scholastic Performance}

\subsection*{Scholastic Performance}

\begin{performance}
\multicolumn{3}{c}{SWARTHMORE COLLEGE} \\
1996 & Introduction to Molecular Biology & A \\
1995 & Introduction to Engineering & A \\
1996 & Introductory Chemistry I & B \\
1995 & Calculus I & A \\
1996 & Calculus II & B \\
1996 & Structures and Design & A \\
1996 & Linear Algebra & B \\
1996 & Physics for Engineers & A \\
1997 & Introductory Chemistry II & C \\
1997 & Organic Chemistry I & A \\
1997 & Structural Materials & B \\
1997 & Structural Materials Laboratory & A \\
1997 & Numerical Computation and Graphics Tools & A \\
1997 & Engineering Graphics and Computer-Assisted Design & A \\
1997 & Principles of Structural Design I & B \\
1997 & Statistics, Probability, and Reliability & A \\
1998 & Principles of Structural Design II & A \\
1999 & Senior Project & A \\
1999 & Biochemistry & A \\
1999 & Cell Biology & A \\
\multicolumn{3}{c}{UC SAN DIEGO} \\
2001 & Seminar in Genetics  & P \\
2002 & Statistics for the Life Sciences & P \\
2003 & Ethics in Biological Research & CRE \\
2004 & Seminar in Physiology and Behavior & P \\
\end{performance}

Except for the scientific ethics course, UC San Diego graduate courses are graded P (pass) or F (fail). Passing is C plus or better. The scientific ethics course is graded CRE (credit) or NC (no credit). Students must attend at least seven of the eight presentation/discussion sessions for credit.


\end{document}
